\documentclass[a4paper,12pt]{article}

\usepackage[italian]{babel}
\usepackage[utf8]{inputenc}
\usepackage[T1]{fontenc}
\usepackage{geometry}
\usepackage{graphicx}
\usepackage{hyperref}
\usepackage{listings}
\usepackage{xcolor}
\usepackage{fancyhdr}
\usepackage{titlesec}
\usepackage{enumitem}
\usepackage{textcomp}

% Configurazione della pagina
\geometry{margin=2.5cm}

% Configurazione dell'header e footer
\pagestyle{fancy}
\fancyhf{}
\renewcommand{\headrulewidth}{0.4pt}
\renewcommand{\footrulewidth}{0.4pt}
\fancyhead[L]{EDU-K.O. - Simulatore di Sopravvivenza Scolastica}
\fancyhead[R]{Esame di Stato 2025}
\fancyfoot[C]{\thepage}

% Configurazione dei titoli
\titleformat{\section}
{\normalfont\Large\bfseries}{\thesection}{1em}{}
\titleformat{\subsection}
{\normalfont\large\bfseries}{\thesubsection}{1em}{}

% Configurazione del codice
\definecolor{codegray}{rgb}{0.95,0.95,0.95}
\definecolor{codegreen}{rgb}{0,0.6,0}
\definecolor{codepurple}{rgb}{0.58,0,0.82}
\definecolor{backcolour}{rgb}{0.95,0.95,0.95}

\lstdefinestyle{mystyle}{
    backgroundcolor=\color{backcolour},   
    commentstyle=\color{codegreen},
    keywordstyle=\color{magenta},
    numberstyle=\tiny\color{codegray},
    stringstyle=\color{codepurple},
    basicstyle=\ttfamily\footnotesize,
    breakatwhitespace=false,         
    breaklines=true,                 
    captionpos=b,                    
    keepspaces=true,                 
    numbers=left,                    
    numbersep=5pt,                  
    showspaces=false,                
    showstringspaces=false,
    showtabs=false,                  
    tabsize=2,
    literate=
        {à}{{\`a}}1 
        {è}{{\`e}}1 
        {ì}{{\`i}}1 
        {ò}{{\`o}}1 
        {ù}{{\`u}}1
        {É}{{\'E}}1
        {È}{{\`E}}1
        {ç}{{\c{c}}}1
        {█}{{$\blacksquare$}}1 
        {♥}{{$\heartsuit$}}1
}

\lstset{style=mystyle}

\begin{document}

\begin{titlepage}
    \centering
    \vspace*{1cm}
    {\Huge\textbf{EDU-K.O.\\}\par}
    \vspace{1cm}
    {\LARGE Il Simulatore di Sopravvivenza Scolastica\par}
    \vspace{2cm}
    {\large Capolavoro per l'Esame di Stato 2025\par}
    \vspace{3cm}
    {\large\itshape Nome Studente: Alessandro Privitera\par}
    {\large\itshape Classe: 5\textsuperscript{a} Informatica\par}
    \vspace{1cm}
    {\large\itshape Istituto Tecnico Industriale "Galileo Ferraris"\par}
    \vspace{1cm}
    {\large Anno Scolastico 2024-2025\par}
\end{titlepage}

\tableofcontents
\newpage

\section{Introduzione}

Questo progetto rappresenta una riflessione critica e provocatoria sul sistema scolastico italiano, tradotta in un videogioco testuale sviluppato in linguaggio C. \textit{EDU-K.O. - Il Simulatore di Sopravvivenza Scolastica} è un'opera che, attraverso la programmazione e l'ironia, mette in luce le contraddizioni e le difficoltà che gli studenti affrontano quotidianamente.

Il programma simula una giornata tipica di uno studente italiano, costretto a navigare tra verifiche a sorpresa, aspettative irrealistiche, e un sistema che spesso sembra progettato per mettere alla prova non solo le conoscenze, ma anche la resistenza psicologica.

\subsection{Motivazione e Obiettivi}

La motivazione alla base di questo progetto nasce dalla mia esperienza personale e da quella condivisa con altri studenti. L'obiettivo non è semplicemente criticare, ma piuttosto:

\begin{itemize}
    \item Stimolare una riflessione critica sul sistema educativo attuale
    \item Evidenziare, attraverso la gamification e l'ironia, problematiche reali che meritano attenzione
    \item Dimostrare come l'informatica possa essere uno strumento di espressione e critica sociale
    \item Proporre implicitamente un sistema educativo più umano ed empatico
\end{itemize}

\subsection{Competenze Dimostrate}

Questo progetto mi ha permesso di mettere in pratica e sviluppare diverse competenze acquisite durante il percorso di studi:

\begin{itemize}
    \item \textbf{Sviluppo Software}: Progettazione e implementazione di un'applicazione interattiva in C
    \item \textbf{Algoritmi e Strutture Dati}: Utilizzo di struct per la gestione dei dati del giocatore
    \item \textbf{Interfacce Utente}: Creazione di un'interfaccia testuale intuitiva
    \item \textbf{Problem Solving}: Gestione delle diverse situazioni di gioco e dei relativi esiti
    \item \textbf{Pensiero Critico}: Analisi del sistema educativo e traduzione delle sue problematiche in meccaniche di gioco
    \item \textbf{Documentazione}: Capacità di presentare e documentare adeguatamente un progetto software
\end{itemize}

\section{Analisi e Progettazione}

\subsection{Analisi del Problema}

Il problema affrontato da questo progetto è duplice:
\begin{enumerate}
    \item Come rappresentare efficacemente, attraverso un programma, le contraddizioni del sistema scolastico
    \item Come creare un'esperienza interattiva che sia al contempo provocatoria ma anche coinvolgente
\end{enumerate}

\subsection{Metodologia di Sviluppo}

Per lo sviluppo di questo progetto ho adottato un approccio iterativo:
\begin{enumerate}
    \item Raccolta di esperienze e situazioni tipiche della vita scolastica
    \item Classificazione di queste situazioni in categorie (mattino, lezioni, verifiche, ecc.)
    \item Progettazione della struttura dati per rappresentare lo stato del giocatore
    \item Implementazione delle singole situazioni di gioco
    \item Testing e bilanciamento delle meccaniche di gioco
    \item Rifinitura dell'interfaccia utente e dei messaggi
\end{enumerate}

\subsection{Strumenti Utilizzati}

\begin{itemize}
    \item \textbf{Linguaggio di Programmazione}: C
    \item \textbf{Ambiente di Sviluppo}: Visual Studio Code
    \item \textbf{Compilatore}: GCC (GNU Compiler Collection)
    \item \textbf{Controllo Versione}: Git
    \item \textbf{Documentazione}: \LaTeX
\end{itemize}

\section{Implementazione}

\subsection{Struttura del Programma}

Il programma è strutturato in diverse sezioni:

\begin{itemize}
    \item Definizione delle strutture dati e costanti
    \item Funzioni di gestione dell'interfaccia utente
    \item Funzioni che simulano le diverse situazioni scolastiche
    \item Funzione principale che coordina il flusso di gioco
\end{itemize}

\subsection{Strutture Dati}

La principale struttura dati utilizzata è la struct \texttt{Studente}, che mantiene lo stato del giocatore:

\begin{lstlisting}[language=C, caption=Struttura dati Studente]
typedef struct {
    char nome[NOME_LENGTH];
    int stress;
    float media;
    int ore_sonno;
    int assenze;
    int note_disciplinari;
    int voglia_vivere;
} Studente;
\end{lstlisting}

Questa struttura permette di tracciare diversi parametri che rappresentano lo stato psicofisico e il rendimento scolastico del giocatore.

\subsection{Flusso di Gioco}

Il gioco si sviluppa attraverso una serie di situazioni che rappresentano momenti tipici della giornata scolastica:

\begin{enumerate}
    \item Risveglio e preparazione mattutina
    \item Lezioni in classe
    \item Verifiche a sorpresa
    \item Colloquio con i docenti
    \item Ritorno a casa e gestione dei compiti
    \item Valutazione finale della giornata
\end{enumerate}

Per ogni situazione, il giocatore deve scegliere tra diverse opzioni, ciascuna con conseguenze diverse sui parametri del personaggio.

\subsection{Meccaniche di Gioco}

Le principali meccaniche di gioco includono:

\begin{itemize}
    \item \textbf{Sistema di scelte}: Ogni situazione presenta al giocatore 3 opzioni
    \item \textbf{Parametri dinamici}: I parametri del giocatore (stress, media, ore di sonno, ecc.) vengono modificati in base alle scelte
    \item \textbf{Conseguenze a catena}: Le scelte fatte in una situazione influenzano le situazioni successive
    \item \textbf{Feedback immediato}: Dopo ogni scelta, il giocatore riceve un feedback sui risultati delle sue azioni
    \item \textbf{Visualizzazione stato}: Dopo ogni situazione, viene mostrato lo stato aggiornato dei parametri
\end{itemize}

\subsection{Interfaccia Utente}

L'interfaccia è minimalista e basata su testo, una scelta deliberata per enfatizzare il contenuto e il messaggio:

\begin{itemize}
    \item Rappresentazione grafica dello stress e della voglia di vivere tramite barre di progresso testuali
    \item Menu di scelta numerici
    \item Messaggi ironici e provocatori che accompagnano ogni situazione
\end{itemize}

\section{Codice Sorgente}

\subsection{Header e Definizioni}

\begin{lstlisting}[language=C, caption=Header e definizioni]
/*
* EDU-K.O. - Il Simulatore di Sopravvivenza Scolastica
* Un viaggio interattivo nell'assurda realtà del sistema educativo italiano
* 
* "Nel codice, come nella vita: lo studente perde sempre"
*/

#include <stdio.h>
#include <stdlib.h>
#include <string.h>
#include <time.h>

// Configurazione del gioco
#define MAX_STRESS 100
#define MAX_VOTO 10
#define NOME_LENGTH 50
\end{lstlisting}

\subsection{Strutture Dati e Array}

\begin{lstlisting}[language=C, caption=Strutture dati e array]
// Struttura per lo studente virtuale
typedef struct {
    char nome[NOME_LENGTH];
    int stress;
    float media;
    int ore_sonno;
    int assenze;
    int note_disciplinari;
    int voglia_vivere;
} Studente;

// Frasi ironiche del sistema
char* frasi_prof[] = {
    "\"Non dovevi studiare solo l'ultima settimana\"",
    "\"Io alla tua età studiavo 12 ore al giorno\"",
    "\"Non ho tempo per spiegazioni individuali\"",
    "\"La creatività non serve, servono le nozioni\"",
    "\"Non è colpa mia se non capisci\"",
    "\"Questo lo avete già fatto in precedenza\"",
    "\"Nel mondo del lavoro non ti passano nulla\"",
    "\"Dovete imparare a essere autonomi\""
};

char* situazioni_assurde[] = {
    "Ti hanno assegnato 8 materie con verifica questa settimana",
    "Il WiFi non funziona ma pretendono che usi il registro elettronico",
    "Ti chiedono di essere creativo ma poi ti bocciano se esci dal programma",
    "Devi fare 30 ore di alternanza scuola-lavoro non retribuite",
    "Ti fanno studiare poesie del 1200 ma non come fare le tasse",
    "La prof cambia le regole del compito durante il compito",
    "Ti chiedono di avere 'spirito critico' ma solo se concordi con loro"
};
\end{lstlisting}

\subsection{Funzione Main}

\begin{lstlisting}[language=C, caption=Funzione main]
int main() {
    srand(time(NULL));
    
    printf("\033[2J\033[1;1H"); // Pulisce lo schermo
    
    intro();
    
    Studente studente;
    inizializza_studente(&studente);
    
    printf("\n\n=== EDU-K.O. - Il Simulatore di Sopravvivenza Scolastica ===\n");
    printf("     \"Benvenuto nell'incubo quotidiano\"               \n\n");
    
    // Varie situazioni di una giornata scolastica
    situazione_mattino(&studente);
    situazione_lezione(&studente);
    situazione_verifica(&studente);
    situazione_colloquio(&studente);
    situazione_casa(&studente);
    
    // Risultato finale
    risultato_finale(&studente);
    
    return 0;
}
\end{lstlisting}

\subsection{Funzioni di Gioco}

Per motivi di spazio, riporto solo alcune delle funzioni principali. Il codice completo è disponibile nell'appendice.

\begin{lstlisting}[language=C, caption=Funzione di inizializzazione]
void inizializza_studente(Studente *s) {
    printf("\nInserisci il tuo nome: ");
    fgets(s->nome, NOME_LENGTH, stdin);
    s->nome[strcspn(s->nome, "\n")] = 0; // Rimuove newline
    
    s->stress = 50;
    s->media = 6.0;
    s->ore_sonno = 5; // Realistico per uno studente italiano
    s->assenze = 0;
    s->note_disciplinari = 0;
    s->voglia_vivere = 50;
    
    printf("\nBene %s, benvenuto nella macchina tritacarne.\n", s->nome);
    pausa();
}
\end{lstlisting}

\begin{lstlisting}[language=C, caption=Funzione di visualizzazione stato]
void mostra_status(Studente *s) {
    printf("\n--- Status di %s ---\n", s->nome);
    printf("Stress: %d/100 ", s->stress);
    for(int i = 0; i < s->stress/10; i++) printf("\n");
    printf("\n");
    
    printf("Media scolastica: %.1f\n", s->media);
    printf("Ore di sonno: %d\n", s->ore_sonno);
    printf("Assenze: %d\n", s->assenze);
    printf("Note disciplinari: %d\n", s->note_disciplinari);
    printf("Voglia di vivere: %d%% ", s->voglia_vivere);
    for(int i = 0; i < s->voglia_vivere/20; i++) printf("♥");
    printf("\n-------------------\n");
}
\end{lstlisting}

\begin{lstlisting}[language=C, caption=Esempio di situazione di gioco]
void situazione_verifica(Studente *s) {
    printf("\n\n=== ORE 10:15 - VERIFICA A SORPRESA DI LATINO ===\n");
    printf("'Ragazzi, compito in classe! Spero abbiate ripassato.'\n");
    printf("Nessuno aveva annunciato nulla. Il panico si diffonde.\n");
    
    printf("\nCome affronti la situazione?\n");
    printf("1. Tenti di copiare dal tuo compagno\n");
    printf("2. Scrivi qualsiasi cosa sperando nella pietà\n");
    printf("3. Consegni in bianco con dignità\n");
    
    int scelta = scelta_utente(1, 3);
    
    switch(scelta) {
        case 1:
            printf("\nLa prof ti becca dopo 2 minuti.\n");
            printf("'%s! VERGOGNATI! Voti annullati per entrambi!'\n", s->nome);
            printf("Il tuo compagno ti odia. Hai perso un amico.\n");
            s->media = (s->media * 4 + 2) / 5;
            s->stress += 35;
            s->voglia_vivere -= 20;
            s->note_disciplinari++;
            break;
            
        case 2:
            printf("\nScrivi frasi con errori grammaticali italiani nella versione latina.\n");
            printf("La prof ti corregge con inchiostro rosso dappertutto.\n");
            printf("Voto: 4. 'Almeno ci hai provato' dice sarcasticamente.\n");
            s->media = (s->media * 4 + 4) / 5;
            s->stress += 25;
            s->voglia_vivere -= 10;
            break;
            
        case 3:
            printf("\nLa prof apprezza l'onestà ma ti mette 2.\n");
            printf("'Non puoi non sapere nulla dopo un anno di latino!'\n");
            printf("(Come se un anno bastasse per imparare una lingua morta)\n");
            s->media = (s->media * 4 + 2) / 5;
            s->stress += 30;
            s->voglia_vivere -= 15;
            break;
    }
    
    mostra_status(s);
    pausa();
}
\end{lstlisting}

\section{Riflessioni e Considerazioni Tecniche}

\subsection{Scelte Implementative}

Nella realizzazione del progetto, ho fatto diverse scelte implementative:

\begin{itemize}
    \item \textbf{Utilizzo di struct}: Per mantenere i dati del giocatore in modo organizzato
    \item \textbf{Passaggio per riferimento}: Per modificare i parametri del giocatore nelle varie funzioni
    \item \textbf{Randomizzazione}: Per rendere ogni partita leggermente diversa (frasi dei professori, situazioni)
    \item \textbf{Interfaccia basata su terminale}: Per mantenere la semplicità e focalizzarsi sul contenuto
\end{itemize}

\subsection{Difficoltà Incontrate}

Durante lo sviluppo del progetto, ho affrontato alcune difficoltà:

\begin{itemize}
    \item Bilanciamento delle meccaniche: trovare il giusto equilibrio tra difficoltà e giocabilità
    \item Gestione dell'input utente: prevenire input non validi e comportamenti inaspettati
    \item Aspetti narrativi: creare situazioni realistiche ma anche provocatorie
\end{itemize}

\subsection{Soluzioni Adottate}

Per risolvere queste difficoltà ho:

\begin{itemize}
    \item Implementato una funzione robusta per la gestione dell'input utente
    \item Testato il programma su diverse piattaforme (Windows, Linux)
    \item Raccolto feedback da altri studenti per calibrare le situazioni di gioco
    \item Utilizzato array di stringhe per gestire facilmente le frasi e le situazioni random
\end{itemize}

\section{Valore Educativo e Sociale}

\subsection{Critica Costruttiva}

Sebbene il progetto adotti un tono ironico e provocatorio, non intende essere una semplice critica fine a se stessa. L'obiettivo è stimolare una riflessione su:

\begin{itemize}
    \item L'impatto psicologico che il sistema attuale può avere sugli studenti
    \item La discrepanza tra gli obiettivi dichiarati dell'educazione e le pratiche quotidiane
    \item La mancanza di attenzione verso il benessere psicologico degli studenti
    \item L'importanza di un approccio più umano e comprensivo nell'educazione
\end{itemize}

\subsection{Possibili Sviluppi}

Questo progetto potrebbe evolversi in diversi modi:

\begin{itemize}
    \item Implementazione di una versione grafica con interfaccia più elaborata
    \item Espansione delle situazioni di gioco per coprire l'intero anno scolastico
    \item Creazione di una versione "utopica" che simuli un sistema educativo ideale
    \item Traduzione in formato web per renderlo accessibile a più persone
    \item Raccolta di dati anonimi per analizzare le scelte più comuni e le situazioni più stressanti
\end{itemize}

\section{Conclusioni}

\textit{EDU-K.O. - Il Simulatore di Sopravvivenza Scolastica} rappresenta un esempio di come la programmazione possa essere utilizzata non solo per creare applicazioni funzionali, ma anche per esprimere idee, stimolare riflessioni e proporre critiche costruttive.

Attraverso questo progetto, ho voluto dimostrare come le competenze tecniche acquisite durante il mio percorso di studi possano essere applicate in modo creativo e significativo. La programmazione diventa così non solo uno strumento tecnico, ma anche un mezzo di espressione e comunicazione.

In un'epoca in cui il sistema educativo è in continua evoluzione, ritengo che sia importante che noi studenti contribuiamo al dibattito con le nostre prospettive e i nostri talenti. Questo progetto è il mio piccolo contributo a questa importante conversazione.

\section{Bibliografia e Riferimenti}

\begin{enumerate}
    \item Kernighan, B. W., \& Ritchie, D. M. (1988). \textit{The C Programming Language}. Prentice Hall.
    \item Ministero dell'Istruzione (2024). \textit{Nota MIM 7557/2024 - Piattaforma UNICA}.
    \item Adams, S. (1996). \textit{The Dilbert Principle}. Harper Business.
    \item Prensky, M. (2001). \textit{Digital Game-Based Learning}. McGraw-Hill.
    \item Robinson, K. (2015). \textit{Creative Schools: The Grassroots Revolution That's Transforming Education}. Viking.
\end{enumerate}

\appendix
\section{Codice Completo}

Il codice completo del progetto è disponibile qua sotto ed è possibile eseguire il programma compilando il seguente file:

\begin{lstlisting}[language=C, caption=Codice completo (file Capolavoro2.c)]
/*
* EDU-K.O. - Il Simulatore di Sopravvivenza Scolastica
* Un viaggio interattivo nell'assurda realtà del sistema educativo italiano
* 
* "Nel codice, come nella vita: lo studente perde sempre"
*/

#include <stdio.h>
#include <stdlib.h>
#include <string.h>
#include <time.h>

// Configurazione del gioco
#define MAX_STRESS 100
#define MAX_VOTO 10
#define NOME_LENGTH 50

// Struttura per lo studente virtuale
typedef struct {
    char nome[NOME_LENGTH];
    int stress;
    float media;
    int ore_sonno;
    int assenze;
    int note_disciplinari;
    int voglia_vivere;
} Studente;

// Frasi ironiche del sistema
char* frasi_prof[] = {
    "\"Non dovevi studiare solo l'ultima settimana\"",
    "\"Io alla tua età studiavo 12 ore al giorno\"",
    "\"Non ho tempo per spiegazioni individuali\"",
    "\"La creatività non serve, servono le nozioni\"",
    "\"Non è colpa mia se non capisci\"",
    "\"Questo lo avete già fatto in precedenza\"",
    "\"Nel mondo del lavoro non ti passano nulla\"",
    "\"Dovete imparare a essere autonomi\""
};

char* situazioni_assurde[] = {
    "Ti hanno assegnato 8 materie con verifica questa settimana",
    "Il WiFi non funziona ma pretendono che usi il registro elettronico",
    "Ti chiedono di essere creativo ma poi ti bocciano se esci dal programma",
    "Devi fare 30 ore di alternanza scuola-lavoro non retribuite",
    "Ti fanno studiare poesie del 1200 ma non come fare le tasse",
    "La prof cambia le regole del compito durante il compito",
    "Ti chiedono di avere 'spirito critico' ma solo se concordi con loro"
};

// Prototipi funzioni
void inizializza_studente(Studente *s);
void mostra_status(Studente *s);
void situazione_mattino(Studente *s);
void situazione_lezione(Studente *s);
void situazione_verifica(Studente *s);
void situazione_colloquio(Studente *s);
void situazione_casa(Studente *s);
void risultato_finale(Studente *s);
void intro();
void pausa();
int scelta_utente(int min, int max);

int main() {
    srand(time(NULL));
    
    printf("\033[2J\033[1;1H"); // Pulisce lo schermo
    
    intro();
    
    Studente studente;
    inizializza_studente(&studente);
    
    printf("\n\n=== EDU-K.O. - Il Simulatore di Sopravvivenza Scolastica ===\n");
    printf("     \"Benvenuto nell'incubo quotidiano\"               \n\n");
    
    // Varie situazioni di una giornata scolastica
    situazione_mattino(&studente);
    situazione_lezione(&studente);
    situazione_verifica(&studente);
    situazione_colloquio(&studente);
    situazione_casa(&studente);
    
    // Risultato finale
    risultato_finale(&studente);
    
    return 0;
}

void inizializza_studente(Studente *s) {
    printf("\nInserisci il tuo nome: ");
    fgets(s->nome, NOME_LENGTH, stdin);
    s->nome[strcspn(s->nome, "\n")] = 0; // Rimuove newline
    
    s->stress = 50;
    s->media = 6.0;
    s->ore_sonno = 5; // Realistico per uno studente italiano
    s->assenze = 0;
    s->note_disciplinari = 0;
    s->voglia_vivere = 50;
    
    printf("\nBene %s, benvenuto nella macchina tritacarne.\n", s->nome);
    pausa();
}

void mostra_status(Studente *s) {
    printf("\n--- Status di %s ---\n", s->nome);
    printf("Stress: %d/100 ", s->stress);
    for(int i = 0; i < s->stress/10; i++) printf("\n");
    printf("\n");
    
    printf("Media scolastica: %.1f\n", s->media);
    printf("Ore di sonno: %d\n", s->ore_sonno);
    printf("Assenze: %d\n", s->assenze);
    printf("Note disciplinari: %d\n", s->note_disciplinari);
    printf("Voglia di vivere: %d%% ", s->voglia_vivere);
    for(int i = 0; i < s->voglia_vivere/20; i++) printf("♥");
    printf("\n-------------------\n");
}

void situazione_mattino(Studente *s) {
    printf("\n\n=== ORE 06:30 - LA SVEGLIA SUONA ===\n");
    printf("Hai dormito %d ore. Ti senti distrutto.\n", s->ore_sonno);
    
    // Situazione assurda random
    printf("\n%s\n", situazioni_assurde[rand() % 7]);
    
    printf("\nCosa fai?\n");
    printf("1. Vai a scuola da zombie\n");
    printf("2. Fingi di essere malato\n");
    printf("3. Piangi nel cuscino per 10 minuti e poi vai\n");
    
    int scelta = scelta_utente(1, 3);
    
    switch(scelta) {
        case 1:
            printf("\nArrivi in classe con le occhiaie fino ai piedi.\n");
            printf("La prof ti guarda e dice: %s\n", frasi_prof[rand() % 8]);
            s->stress += 15;
            s->voglia_vivere -= 10;
            break;
            
        case 2:
            printf("\nTua madre non ci crede e ti obbliga ad andare.\n");
            printf("Arrivi in ritardo e prendi una nota.\n");
            s->note_disciplinari++;
            s->stress += 20;
s->assenze++; // Contano come assenza anche se vai
            break;
            
        case 3:
            printf("\nArrivi con gli occhi rossi. Il bidello ti chiede se va tutto bene.\n");
            printf("È l'unica persona che si preoccupa per te oggi.\n");
            s->stress += 10;
            s->voglia_vivere -= 5;
            break;
    }
    
    mostra_status(s);
    pausa();
}

void situazione_lezione(Studente *s) {
    printf("\n\n=== ORE 08:15 - PRIMA ORA: MATEMATICA ===\n");
    printf("La prof entra e dice: 'Spero abbiate studiato il capitolo 7'.\n");
    printf("Tu ovviamente hai studiato il capitolo 6.\n");
    
    printf("\nCosa fai?\n");
    printf("1. Confessi di aver studiato il capitolo sbagliato\n");
    printf("2. Fingi di sapere e speri di non essere interrogato\n");
    printf("3. Fingi un malore e vai in infermeria\n");
    
    int scelta = scelta_utente(1, 3);
    
    switch(scelta) {
        case 1:
            printf("\nLa prof: 'COME È POSSIBILE? È SUL REGISTRO DA DUE SETTIMANE!'\n");
            printf("Vieni interrogato comunque e prendi 3.\n");
            s->media = (s->media * 4 + 3) / 5;
            s->stress += 25;
            s->voglia_vivere -= 15;
            break;
            
        case 2:
            printf("\nViene chiamato il tuo compagno di banco.\n");
            printf("Mentre lui balbetta le risposte, la prof nota che stai leggendo sotto il banco.\n");
            printf("Nota disciplinare per 'comportamento scorretto'.\n");
            s->note_disciplinari++;
            s->stress += 20;
            break;
            
        case 3:
            printf("\nIn infermeria non c'è nessuno (come al solito).\n");
            printf("Ti siedi e apri il libro di matematica, ma è tardi per recuperare.\n");
            s->stress += 15;
            s->ore_sonno--; // Lo stress ti sta consumando
            break;
    }
    
    mostra_status(s);
    pausa();
}

void situazione_verifica(Studente *s) {
    printf("\n\n=== ORE 10:15 - VERIFICA A SORPRESA DI LATINO ===\n");
    printf("'Ragazzi, compito in classe! Spero abbiate ripassato.'\n");
    printf("Nessuno aveva annunciato nulla. Il panico si diffonde.\n");
    
    printf("\nCome affronti la situazione?\n");
    printf("1. Tenti di copiare dal tuo compagno\n");
    printf("2. Scrivi qualsiasi cosa sperando nella pietà\n");
    printf("3. Consegni in bianco con dignità\n");
    
    int scelta = scelta_utente(1, 3);
    
    switch(scelta) {
        case 1:
            printf("\nLa prof ti becca dopo 2 minuti.\n");
            printf("'%s! VERGOGNATI! Voti annullati per entrambi!'\n", s->nome);
            printf("Il tuo compagno ti odia. Hai perso un amico.\n");
            s->media = (s->media * 4 + 2) / 5;
            s->stress += 35;
            s->voglia_vivere -= 20;
            s->note_disciplinari++;
            break;
            
        case 2:
            printf("\nScrivi frasi con errori grammaticali italiani nella versione latina.\n");
            printf("La prof ti corregge con inchiostro rosso dappertutto.\n");
            printf("Voto: 4. 'Almeno ci hai provato' dice sarcasticamente.\n");
            s->media = (s->media * 4 + 4) / 5;
            s->stress += 25;
            s->voglia_vivere -= 10;
            break;
            
        case 3:
            printf("\nLa prof apprezza l'onestà ma ti mette 2.\n");
            printf("'Non puoi non sapere nulla dopo un anno di latino!'\n");
            printf("(Come se un anno bastasse per imparare una lingua morta)\n");
            s->media = (s->media * 4 + 2) / 5;
            s->stress += 30;
            s->voglia_vivere -= 15;
            break;
    }
    
    mostra_status(s);
    pausa();
}

void situazione_colloquio(Studente *s) {
    printf("\n\n=== ORE 13:00 - COLLOQUIO CON LA COORDINATRICE ===\n");
    printf("'%s, vieni un momento. Dobbiamo parlare.'\n", s->nome);
    printf("Il cuore ti si ferma. Cosa ho fatto adesso?\n");
    
    printf("\nCome ti presenti?\n");
    printf("1. Umile e pentito (anche se non sai di cosa)\n");
    printf("2. Sicuro di te e pronto al confronto\n");
    printf("3. Rassegnato al destino\n");
    
    int scelta = scelta_utente(1, 3);
    
    switch(scelta) {
        case 1:
            printf("\nProf: 'Ho notato un calo nel tuo rendimento.'\n");
            printf("Tu: 'Ha ragione prof, mi impegnerò di più.'\n");
            printf("Prof: 'Bene, voglio vedere miglioramenti o chiamo i tuoi genitori.'\n");
            s->stress += 20;
            s->voglia_vivere -= 10;
            break;
            
        case 2:
            printf("\nTu: 'Se c'è un problema, parliamone apertamente.'\n");
            printf("Prof: 'Il problema è il tuo atteggiamento! Non ti rendi conto della gravità!'\n");
            printf("Escalation. Nota disciplinare. I tuoi genitori vengono avvisati.\n");
            s->note_disciplinari++;
            s->stress += 35;
            s->voglia_vivere -= 20;
            break;
            
        case 3:
            printf("\nTu: 'Lo so, sono un fallimento.'\n");
            printf("Prof: 'Non dire così... ma effettivamente devi reagire.'\n");
            printf("Ti senti peggio di prima. Almeno non hai preso una nota.\n");
            s->stress += 15;
            s->voglia_vivere -= 15;
            break;
    }
    
    mostra_status(s);
    pausa();
}

void situazione_casa(Studente *s) {
    printf("\n\n=== ORE 15:30 - FINALMENTE A CASA ===\n");
    printf("Guardi l'agenda: 3 compiti scritti, 4 capitoli da studiare, ricerca di gruppo.\n");
    printf("Domani interrogazioni in 2 materie.\n");
    
    printf("\nCome organizzi il pomeriggio?\n");
    printf("1. Studio 6 ore filate per recuperare\n");
    printf("2. Fai una pausa e poi studi (rischi di non finire)\n");
    printf("3. Ti arrendi e guardi Netflix\n");
    
    int scelta = scelta_utente(1, 3);
    
    switch(scelta) {
        case 1:
            printf("\nStudi fino alle 21:30. Riesci a fare quasi tutto.\n");
            printf("Ma hai mal di testa e non hai cenato bene.\n");
            printf("Ti addormenti alle 01:00. Domani saranno 5 ore di sonno.\n");
            s->stress += 25;
            s->ore_sonno = 5;
            s->voglia_vivere -= 15;
            s->media += 0.2; // Piccolo miglioramento nella media
            break;
            
        case 2:
            printf("\nTi riposi 1 ora, poi cominci a studiare.\n");
            printf("Alle 23:00 ti accorgi che hai fatto solo metà.\n");
            printf("Panico. Studi fino alle 02:00 in ansia.\n");
            s->stress += 35;
            s->ore_sonno = 4;
            s->voglia_vivere -= 20;
            break;
            
        case 3:
            printf("\nGuardi una serie fino a tardi. Ti senti in colpa.\n");
            printf("Domani sarà un disastro, ma almeno stasera respiri.\n");
            printf("Ansia da domani: già pensi alle scuse per i prof.\n");
            s->stress += 40;
            s->ore_sonno = 6;
            s->voglia_vivere -= 10;
            s->media -= 0.3; // Peggioramento della media
            break;
    }
    
    mostra_status(s);
    pausa();
}

void risultato_finale(Studente *s) {
    printf("\n\n=== FINE GIORNATA - RIEPILOGO ESISTENZIALE ===\n");
    
    printf("\n STATISTICHE DI %s:\n", s->nome);
    printf("----------------------------------------\n");
    mostra_status(s);
    
    printf("\n ANALISI PSICOLOGICA:\n");
    if(s->stress >= 90) {
        printf("Sei vicino al burnout. Il sistema ha quasi vinto.\n");
    } else if(s->stress >= 70) {
        printf("Il peso del sistema ti sta schiacciando.\n");
    } else if(s->stress >= 50) {
        printf("Sopravvivi, ma a quale prezzo?\n");
    } else {
        printf("Hai resistito, ma domani si ricomincia.\n");
    }
    
    printf("\n REPORT GIORNALIERO:\n");
    printf("- Media scolastica: %.1f (", s->media);
    if(s->media >= 6.0) printf("Sufficiente, ma a che costo?)\n");
    else printf("Insufficiente, come prevedibile)\n");
    
    printf("- Ore di sonno: %d (", s->ore_sonno);
    if(s->ore_sonno >= 8) printf("Miracolo!)\n");
    else if(s->ore_sonno >= 6) printf("Almeno qualcosa)\n");
    else printf("Zombie mode attivo)\n");
    
    printf("- Stress accumulato: %d/100\n", s->stress);
    printf("- Voglia di vivere: %d%%\n", s->voglia_vivere);
    
    printf("\n CONCLUSIONE:\n");
    printf("----------------------------------------\n");
    printf("Un'altra giornata di sopravvivenza completata.\n");
    printf("La scuola ti prepara alla vita... \n");
    printf("...ti prepara a soffrire.\n\n");
    
    printf("Il sistema non è fatto per lo studente,\n");
    printf("lo studente è fatto per il sistema.\n\n");
    
    printf("\"Nel mondo del lavoro sarà uguale\" dicono.\n");
    printf("E se fosse il momento di cambiare tutto?\n\n");
    
    printf("Grazie per aver giocato a EDU-K.O.\n");
    printf("Domani, stessa ora, stessa sofferenza.\n");
    printf("----------------------------------------\n");
}

void intro() {
    printf("\n");
    printf("\n");
    printf("           Il Simulatore di Sopravvivenza Scolastica\n");
    printf("          \"Benvenuto nel sistema che vuole distruggerti\"\n");
}

void pausa() {
    printf("\nPremi INVIO per continuare...");
    getchar();
}

int scelta_utente(int min, int max) {
    int scelta;
    char buffer[100];
    
    while(1) {
        printf("\nScelta (inserisci il numero): ");
        if(fgets(buffer, sizeof(buffer), stdin) != NULL) {
            scelta = atoi(buffer);
            if(scelta >= min && scelta <= max) {
                return scelta;
            }
        }
        printf("Scelta non valida. Riprova.\n");
    }
}
\end{lstlisting}

\section{Istruzioni per la Compilazione ed Esecuzione}

Per compilare ed eseguire il programma, seguire questi passaggi:

\begin{enumerate}
    \item Aprire un terminale
    \item Navigare nella directory contenente il file sorgente
    \item Eseguire il comando di compilazione: \texttt{gcc -o edu-ko Capolavoro2.c}
    \item Eseguire il programma: \texttt{./edu-ko}
\end{enumerate}

\end{document}